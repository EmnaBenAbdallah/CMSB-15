
%%%%%%%%%%%%%%%%%%%%%%% file articleCAV.tex %%%%%%%%%%%%%%%%%%%%%%%%%
%
% Article à déposer pour CMSB 2015
%
%%%%%%%%%%%%%%%%%%%%%%%%%%%%%%%%%%%%%%%%%%%%%%%%%%%%%%%%%%%%%%%%%%%


\documentclass[runningheads,a4paper]{llncs}

\usepackage{amssymb}
\setcounter{tocdepth}{3}
\usepackage{graphicx}

%%%%%%%%%%%%%%%%%%%%% Packages %%%%%%%%%%%%%%%%%%%%%%%%%%%%%%
\usepackage{hyperref}


\usepackage{amsmath}  % Maths
\usepackage{amsfonts} % Maths
\usepackage{amssymb}  % Maths
\usepackage{stmaryrd} % Maths (crochets doubles)


%%%%%%%%%%%%%%%%%%%%
\usepackage{listings}
% Définition du langage ASP
\lstdefinelanguage{ASP}{\^^M}
}

% Définition des styles de tous les listings du document
\lstset{language=ASP,
basicstyle=\small,
columns=flexible,
keywordstyle=\bfseries,
firstnumber=last
}
\renewcommand{\thelstnumber}{\the\value{lstnumber}}
%% fin définition

%%%%%%%%%%%%%%%%%%%%%%%%%%%%%%%%%%
\usepackage{enumerate} % Personnalisation de la numérotation des listes
\usepackage{url}     % Mise en forme + liens pour URLs
\usepackage{array}   % Tableaux évolués
\usepackage{comment}

\usepackage{prettyref}
\newrefformat{def}{Definition~\ref{#1}}
\newrefformat{fig}{Figure~\ref{#1}}
\newrefformat{pro}{Property~\ref{#1}}
\newrefformat{pps}{Proposition~\ref{#1}}
\newrefformat{lem}{Lemma~\ref{#1}}
\newrefformat{th}{Theorem~\ref{#1}}
\newrefformat{sec}{Section~\ref{#1}}
%\newrefformat{subsec}{Subsect.~\ref{#1}}
\newrefformat{suppl}{Appendix~\ref{#1}}
\newrefformat{eq}{Eq.~\eqref{#1}}
\def\pref{\prettyref}


\usepackage{tikz}
\newdimen\pgfex
\newdimen\pgfem
\usetikzlibrary{arrows,shapes,shadows,scopes}
\usetikzlibrary{positioning}
\usetikzlibrary{matrix}
\usetikzlibrary{decorations.text}
\usetikzlibrary{decorations.pathmorphing}

%% Macros relatives à la traduction de PH avec arcs neutralisants vers PH à k-priorités fixes

% Macros générales
%\newcommand{\ie}{\textit{i.e.} }
\newcommand{\segm}[2]{\llbracket #1; #2 \rrbracket}
%\newcommand{\f}[1]{\mathsf{#1}}

% Notations générales pour PH
\newcommand{\PH}{\mathcal{PH}}
%\newcommand{\PHs}{\mathcal{S}}
\newcommand{\PHs}{\Sigma}
%\newcommand{\PHp}{\mathcal{P}}
\newcommand{\PHp}{\textcolor{red}{\mathcal{P}}}
%\newcommand{\PHproc}{\mathcal{P}}
\newcommand{\PHproc}{\mathbf{Proc}}
\newcommand{\Proc}{\PHproc}
\newcommand{\PHh}{\mathcal{H}}
\newcommand{\PHa}{\PHh}
%\newcommand{\PHa}{\mathcal{A}}
\newcommand{\PHl}{\mathcal{L}}
\newcommand{\PHn}{\mathcal{N}}

\newcommand{\PHhitter}{\mathsf{hitter}}
\newcommand{\PHtarget}{\mathsf{target}}
\newcommand{\PHbounce}{\mathsf{bounce}}
%\newcommand{\PHsort}{\Sigma}
\newcommand{\PHsort}{\PHs}

%\newcommand{\PHfrappeur}{\mathsf{frappeur}}
%\newcommand{\PHcible}{\mathsf{cible}}
%\newcommand{\PHbond}{\mathsf{bond}}
%\newcommand{\PHsorte}{\mathsf{sorte}}
%\newcommand{\PHbloquant}{\mathsf{bloquante}}
%\newcommand{\PHbloque}{\mathsf{bloquee}}

%\newcommand{\PHfrappeR}{\textcolor{red}{\rightarrow}}
%\newcommand{\PHmonte}{\textcolor{red}{\Rsh}}

\newcommand{\PHhitA}{\rightarrow}
\newcommand{\PHhitB}{\Rsh}
%\newcommand{\PHfrappe}[3]{\mbox{$#1\PHhitA#2\PHhitB#3$}}
%\newcommand{\PHfrappebond}[2]{\mbox{$#1\PHhitB#2$}}
\newcommand{\PHhit}[3]{#1\PHhitA#2\PHhitB#3}
\newcommand{\PHfrappe}{\PHhit}
\newcommand{\PHhbounce}[2]{#1\PHhitB#2}
\newcommand{\PHobj}[2]{\mbox{$#1\PHhitB^*\!#2$}}
\newcommand{\PHobjectif}{\PHobj}
\newcommand{\PHconcat}{::}
%\newcommand{\PHneutralise}{\rtimes}

% Actions plurielles
\newcommand{\PHhitmultsymbol}{\rightarrowtail}
\newcommand{\PHhitmult}[2]{\mbox{$#1 \PHhitmultsymbol #2$}}
\newcommand{\PHfrappemult}{\PHhitmult}
\newcommand{\PHfrappemults}[2]{\PHhitmult{\{#1\}}{\{#2\}}}

\def\PHget#1#2{{#1[#2]}}
%\newcommand{\PHchange}[2]{#1\langle #2 \rangle}
%\newcommand{\PHchange}[2]{(#1 \Lleftarrow #2)}
%\newcommand{\PHarcn}[2]{\mbox{$#1\PHneutralise#2$}}
\newcommand{\PHplay}{\cdot}

\newcommand{\PHstate}[1]{\mbox{$\langle #1 \rangle$}}
\newcommand{\PHetat}{\PHstate}

\def\supp{\mathsf{support}}
\def\first{\mathsf{first}}
\def\last{\mathsf{last}}

\def\DNtrans{\rightarrow_{ADN}}
\def\DNdef{(\mathbb F, \langle f^1, \dots, f^n\rangle)}
\def\DNdep{\mathsf{dep}}
\def\PHPtrans{\rightarrow_{PH}}
\def\get#1#2{#1[{#2}]}
\def\encodeF#1{\mathbf{#1}}
\def\toPH{\encodeF{PH}}
\def\card#1{|#1|}
\def\decode#1{\llbracket#1\rrbracket}
\def\encode#1{\llparenthesis#1\rrparenthesis}
\def\Hits{\PHa}
\def\hit{\PHhit}
\def\play{\cdot}

\def\Pint{\textsc{PINT}}




%%%%%
% Macros générales
\def\Pint{\textsc{PINT}}


% Notations spécifiques à ce papier
\newcommand{\PHdirectpredec}[1]{\PHs^{-1}(#1)}
\newcommand{\PHpredec}[1]{\f{pred}(#1)}
\newcommand{\PHpredecgene}[1]{\f{reg}({#1})}
\newcommand{\PHpredeccs}[1]{\PHpredec{#1} \setminus \Gamma}

\tikzstyle{boxed ph}=[]
\tikzstyle{sort}=[fill=lightgray,rounded corners]
\tikzstyle{process}=[circle,draw,minimum size=15pt,fill=white,
font=\footnotesize,inner sep=1pt]
\tikzstyle{black process}=[process, fill=black,text=white, font=\bfseries]
\tikzstyle{gray process}=[process, draw=black, fill=lightgray]
\tikzstyle{current process}=[process, draw=black, fill=lightgray]
\tikzstyle{process box}=[white,draw=black,rounded corners]
\tikzstyle{tick label}=[font=\footnotesize]
\tikzstyle{tick}=[black,-]%,densely dotted]
\tikzstyle{hit}=[->,>=angle 45]
\tikzstyle{selfhit}=[min distance=30pt,curve to]
\tikzstyle{bounce}=[densely dotted,->,>=latex]
\tikzstyle{hl}=[font=\bfseries,very thick]
\tikzstyle{hl2}=[hl]
\tikzstyle{nohl}=[font=\normalfont,thin]


\tikzstyle{aS}=[every edge/.style={draw,->,>=stealth}]
\tikzstyle{Asol}=[draw,circle,minimum size=5pt,inner sep=0,node distance=1.5cm]
\tikzstyle{Aproc}=[draw,node distance=1.2cm]
\tikzstyle{Aobj}=[node distance=1.5cm]
\tikzstyle{Anos}=[font=\Large]

%\tikzstyle{AprocPrio}=[Aproc,double]
\tikzstyle{AsolPrio}=[Asol,double]
\tikzstyle{AprocPrio}=[Aproc,double]
\tikzstyle{aSPrio}=[aS,double]

% Commandes À FAIRE
%\usepackage{color} % Couleurs du texte
%\newcommand{\todo}[1]{\textcolor{red}{\textbf{[[#1]]}}}
%\newcommand{\TODO}{\todo{TODO}}

%%%%%
% Id est
%\newcommand{\ie}{\textit{i.e.} }
\newcommand{\ie}{i.e.\ }
\newcommand{\resp}{resp.\ }

% Césures
\hyphenation{pa-ra-me-tri-za-tion}
\hyphenation{pa-ra-me-tri-za-tions}

% Macros relatives à la traduction de PH avec arcs neutralisants vers PH à k-priorités fixes

% Macros générales
%\newcommand{\ie}{\textit{i.e.} }
\newcommand{\segm}[2]{\llbracket #1; #2 \rrbracket}
%\newcommand{\f}[1]{\mathsf{#1}}

% Notations générales pour PH
\newcommand{\PH}{\mathcal{PH}}
%\newcommand{\PHs}{\mathcal{S}}
\newcommand{\PHs}{\Sigma}
%\newcommand{\PHp}{\mathcal{P}}
\newcommand{\PHp}{\textcolor{red}{\mathcal{P}}}
%\newcommand{\PHproc}{\mathcal{P}}
\newcommand{\PHproc}{\mathbf{Proc}}
\newcommand{\Proc}{\PHproc}
\newcommand{\PHh}{\mathcal{H}}
\newcommand{\PHa}{\PHh}
%\newcommand{\PHa}{\mathcal{A}}
\newcommand{\PHl}{\mathcal{L}}
\newcommand{\PHn}{\mathcal{N}}

\newcommand{\PHhitter}{\mathsf{hitter}}
\newcommand{\PHtarget}{\mathsf{target}}
\newcommand{\PHbounce}{\mathsf{bounce}}
%\newcommand{\PHsort}{\Sigma}
\newcommand{\PHsort}{\PHs}

%\newcommand{\PHfrappeur}{\mathsf{frappeur}}
%\newcommand{\PHcible}{\mathsf{cible}}
%\newcommand{\PHbond}{\mathsf{bond}}
%\newcommand{\PHsorte}{\mathsf{sorte}}
%\newcommand{\PHbloquant}{\mathsf{bloquante}}
%\newcommand{\PHbloque}{\mathsf{bloquee}}

%\newcommand{\PHfrappeR}{\textcolor{red}{\rightarrow}}
%\newcommand{\PHmonte}{\textcolor{red}{\Rsh}}

\newcommand{\PHhitA}{\rightarrow}
\newcommand{\PHhitB}{\Rsh}
%\newcommand{\PHfrappe}[3]{\mbox{$#1\PHhitA#2\PHhitB#3$}}
%\newcommand{\PHfrappebond}[2]{\mbox{$#1\PHhitB#2$}}
\newcommand{\PHhit}[3]{#1\PHhitA#2\PHhitB#3}
\newcommand{\PHfrappe}{\PHhit}
\newcommand{\PHhbounce}[2]{#1\PHhitB#2}
\newcommand{\PHobj}[2]{\mbox{$#1\PHhitB^*\!#2$}}
\newcommand{\PHobjectif}{\PHobj}
\newcommand{\PHconcat}{::}
%\newcommand{\PHneutralise}{\rtimes}

% Actions plurielles
\newcommand{\PHhitmultsymbol}{\rightarrowtail}
\newcommand{\PHhitmult}[2]{\mbox{$#1 \PHhitmultsymbol #2$}}
\newcommand{\PHfrappemult}{\PHhitmult}
\newcommand{\PHfrappemults}[2]{\PHhitmult{\{#1\}}{\{#2\}}}

\def\PHget#1#2{{#1[#2]}}
%\newcommand{\PHchange}[2]{#1\langle #2 \rangle}
%\newcommand{\PHchange}[2]{(#1 \Lleftarrow #2)}
%\newcommand{\PHarcn}[2]{\mbox{$#1\PHneutralise#2$}}
\newcommand{\PHplay}{\cdot}

\newcommand{\PHstate}[1]{\mbox{$\langle #1 \rangle$}}
\newcommand{\PHetat}{\PHstate}

\def\supp{\mathsf{support}}
\def\first{\mathsf{first}}
\def\last{\mathsf{last}}

\def\DNtrans{\rightarrow_{ADN}}
\def\DNdef{(\mathbb F, \langle f^1, \dots, f^n\rangle)}
\def\DNdep{\mathsf{dep}}
\def\PHPtrans{\rightarrow_{PH}}
\def\get#1#2{#1[{#2}]}
\def\encodeF#1{\mathbf{#1}}
\def\toPH{\encodeF{PH}}
\def\card#1{|#1|}
\def\decode#1{\llbracket#1\rrbracket}
\def\encode#1{\llparenthesis#1\rrparenthesis}
\def\Hits{\PHa}
\def\hit{\PHhit}
\def\play{\cdot}

\def\Pint{\textsc{PINT}}



\usepackage{ifthen}

\newcommand{\currentScope}{}
\newcommand{\currentSort}{}
\newcommand{\currentSortLabel}{}
\newcommand{\currentAlign}{}
\newcommand{\currentSize}{}

\newcounter{la}
\newcommand{\TSetSortLabel}[2]{
  \expandafter\repcommand\expandafter{\csname TUserSort@#1\endcsname}{#2}
}
\newcommand{\TSort}[4]{
  \renewcommand{\currentScope}{#1}
  \renewcommand{\currentSort}{#2}
  \renewcommand{\currentSize}{#3}
  \renewcommand{\currentAlign}{#4}
  \ifcsname TUserSort@\currentSort\endcsname
    \renewcommand{\currentSortLabel}{\csname TUserSort@\currentSort\endcsname}
  \else
    \renewcommand{\currentSortLabel}{\currentSort}
  \fi
  \begin{scope}[shift={\currentScope}]
  \ifthenelse{\equal{\currentAlign}{l}}{
    \filldraw[process box] (-0.5,-0.5) rectangle (0.5,\currentSize-0.5);
    \node[sort] at (-0.2,\currentSize-0.4) {\currentSortLabel};
   }{\ifthenelse{\equal{\currentAlign}{r}}{
     \filldraw[process box] (-0.5,-0.5) rectangle (0.5,\currentSize-0.5);
     \node[sort] at (0.2,\currentSize-0.4) {\currentSortLabel};
   }{
    \filldraw[process box] (-0.5,-0.5) rectangle (\currentSize-0.5,0.5);
    \ifthenelse{\equal{\currentAlign}{t}}{
      \node[sort,anchor=east] at (-0.3,0.2) {\currentSortLabel};
    }{
      \node[sort] at (-0.6,-0.2) {\currentSortLabel};
    }
   }}
  \setcounter{la}{\currentSize}
  \addtocounter{la}{-1}
  \foreach \i in {0,...,\value{la}} {
    \TProc{\i}
  }
  \end{scope}
}

\newcommand{\TTickProc}[2]{ % pos, label
  \ifthenelse{\equal{\currentAlign}{l}}{
    \draw[tick] (-0.6,#1) -- (-0.4,#1);
    \node[tick label, anchor=east] at (-0.55,#1) {#2};
   }{\ifthenelse{\equal{\currentAlign}{r}}{
    \draw[tick] (0.6,#1) -- (0.4,#1);
    \node[tick label, anchor=west] at (0.55,#1) {#2};
   }{
    \ifthenelse{\equal{\currentAlign}{t}}{
      \draw[tick] (#1,0.6) -- (#1,0.4);
      \node[tick label, anchor=south] at (#1,0.55) {#2};
    }{
      \draw[tick] (#1,-0.6) -- (#1,-0.4);
      \node[tick label, anchor=north] at (#1,-0.55) {#2};
    }
   }}
}
\newcommand{\TSetTick}[3]{
  \expandafter\repcommand\expandafter{\csname TUserTick@#1_#2\endcsname}{#3}
}

\newcommand{\myProc}[3]{
  \ifcsname TUserTick@\currentSort_#1\endcsname
    \TTickProc{#1}{\csname TUserTick@\currentSort_#1\endcsname}
  \else
    \TTickProc{#1}{#1}
  \fi
  \ifthenelse{\equal{\currentAlign}{l}\or\equal{\currentAlign}{r}}{
    \node[#2] (\currentSort_#1) at (0,#1) {#3};
  }{
    \node[#2] (\currentSort_#1) at (#1,0) {#3};
  }
}
\newcommand{\TSetProcStyle}[2]{
  \expandafter\repcommand\expandafter{\csname TUserProcStyle@#1\endcsname}{#2}
}
\newcommand{\TProc}[1]{
  \ifcsname TUserProcStyle@\currentSort_#1\endcsname
    \myProc{#1}{\csname TUserProcStyle@\currentSort_#1\endcsname}{}
  \else
    \myProc{#1}{process}{}
  \fi
}

\newcommand{\repcommand}[2]{
  \providecommand{#1}{#2}
  \renewcommand{#1}{#2}
}
\newcommand{\THit}[5]{
  \path[hit] (#1) edge[#2] (#3#4);
  \expandafter\repcommand\expandafter{\csname TBounce@#3@#5\endcsname}{#4}
}
\newcommand{\TBounce}[4]{
  (#1\csname TBounce@#1@#3\endcsname) edge[#2] (#3#4)
}

%\newcommand{\TState}[1]{
%  \foreach \proc in {#1} {
%    \node[current process] (\proc) at (\proc.center) {};
%  }
%}

\newcommand{\TState}[1]{
  \foreach \proc in {#1} {
        \node[current process] (\proc) at (\proc.center) {};
  };
}
\newcommand{\TCoopHit}[6]{
  \node[#2, apdot] at (#3) {};
  \foreach \proc in {#1} {
    \draw[#2,-] (#3) edge (\proc);
  }
  \path[hit] (#3) edge[#2] (#4#5);
  \expandafter\repcommand\expandafter{\csname TBounce@#4@#6\endcsname}{#5}
}

% ex : \TAction{c_1}{a_1.west}{a_0.north west}{}{right}
% #1 = frappeur
% #2 = cible
% #3 = bond
% #4 = style frappe
% #5 = style bond
\newcommand{\TAction}[5]{
  \THit{#1}{#4}{#2}{}{#3}
  \path[bounce, bend #5=50] \TBounce{#2}{}{#3}{};
}

% ex : \TActionPlur{f_1, c_0}{a_0.west}{a_1.south west}{}{3.5,2.5}{left}
% #1 = frappeur
% #2 = cible
% #3 = bond
% #4 = style frappe
% #5 = coordonnées point central
% #6 = direction bond
\newcommand{\TActionPlur}[6]{
  \TCoopHit{#1}{#4}{#5}{#2}{}{#3}
  \path[bounce, bend #6=50] \TBounce{#2}{}{#3}{};
}

% Styles TikZ et couleurs personnalisées

\usepackage{tikz}

\newdimen\pgfex
\newdimen\pgfem
\usetikzlibrary{arrows,shapes,shadows,scopes}
\usetikzlibrary{positioning}
\usetikzlibrary{matrix}
\usetikzlibrary{decorations.text}
\usetikzlibrary{decorations.pathmorphing}
\usetikzlibrary{arrows,shapes}

\definecolor{lightgray}{rgb}{0.8,0.8,0.8}
\definecolor{lightgrey}{rgb}{0.8,0.8,0.8}

\definecolor{lightred}{rgb}{1,0.8,0.8}
\definecolor{lightgreen}{rgb}{0.7,1,0.7}
\definecolor{darkgreen}{rgb}{0,0.5,0}
\definecolor{darkblue}{rgb}{0,0,0.5}
\definecolor{darkyellow}{rgb}{0.5,0.5,0}
\definecolor{lightyellow}{rgb}{1,1,0.6}
\definecolor{darkcyan}{rgb}{0,0.6,0.6}
\definecolor{lightcyan}{rgb}{0.6,1,1}
\definecolor{darkorange}{rgb}{0.8,0.2,0}
\definecolor{notsodarkred}{rgb}{0.8,0,0}

\definecolor{notsodarkgreen}{rgb}{0,0.7,0}

%\definecolor{coloract}{rgb}{0,1,0}
%\definecolor{colorinh}{rgb}{1,0,0}
\colorlet{coloract}{darkgreen}
\colorlet{colorinh}{red}
\colorlet{coloractgray}{lightgreen}
\colorlet{colorinhgray}{lightred}
\colorlet{colorinf}{darkgray}
\colorlet{coloractgray}{lightgreen}
\colorlet{colorinhgray}{lightred}

\colorlet{colorgray}{lightgray}
\colorlet{colorhl}{blue}


\tikzstyle{boxed ph}=[]
\tikzstyle{sort}=[fill=lightgray, rounded corners, draw=black]
\tikzstyle{process}=[circle,draw,minimum size=15pt,fill=white,font=\footnotesize,inner sep=1pt]
%\tikzstyle{black process}=[process, draw=blue, fill=red,text=black,font=\bfseries]
\tikzstyle{gray process}=[process, draw=black, fill=lightgray]
\tikzstyle{highlighted process}=[current process, fill=gray]
\tikzstyle{process box}=[fill=none,draw=black,rounded corners]
%\tikzstyle{current process}=[process, draw=black, fill=lightgray]
\tikzstyle{current process}=[process,fill=lightgray]
\tikzstyle{hl process}=[process,fill=blue!30]
\tikzstyle{tick label}=[font=\footnotesize]
\tikzstyle{tick}=[densely dotted] %-
\tikzstyle{hit}=[->,>=angle 45]
\tikzstyle{selfhit}=[min distance=50pt,curve to]
\tikzstyle{bounce}=[densely dotted,>=stealth',->]
\tikzstyle{ulhit}=[draw=lightgray,fill=lightgray]
\tikzstyle{pulhit}=[fill=lightgray]
\tikzstyle{bulhit}=[draw=lightgray]
\tikzstyle{hl}=[very thick,colorhl]
\tikzstyle{hlb}=[very thick]
\tikzstyle{hlhit}=[hl]
%\tikzstyle{hl2}=[hl]
%\tikzstyle{nohl}=[font=\normalfont,thin]

\tikzstyle{update}=[draw,->,dashed,shorten >=.7cm,shorten <=.7cm]

\tikzstyle{unprio}=[draw,thin]%[double]
%\tikzstyle{prio}=[draw,thick,-stealth]%[double]
\tikzstyle{prio}=[draw,-stealth,double]

\tikzstyle{hitless graph}=[every edge/.style={draw=red,-}]

\tikzstyle{aS}=[every edge/.style={draw,->,>=stealth}]
\tikzstyle{Asol}=[draw,circle,minimum size=5pt,inner sep=0,node distance=1cm]
\tikzstyle{Aproc}=[draw,node distance=1.2cm]
\tikzstyle{Aobj}=[node distance=1.5cm]
\tikzstyle{Anos}=[font=\Large]

\tikzstyle{AsolPrio}=[Asol,double]
\tikzstyle{AprocPrio}=[Aproc,double]
\tikzstyle{aSPrio}=[aS,double]

\colorlet{colorhlwarn}{notsodarkred}
\colorlet{colorhlwarnbg}{lightred}
\tikzstyle{Ahl}=[very thick,fill=colorhlwarnbg,draw=colorhlwarn,text=colorhlwarn]
\tikzstyle{Ahledge}=[very thick,double=colorhlwarnbg,draw=colorhlwarn,color=colorhlwarn]





%\definecolor{darkred}{rgb}{0.5,0,0}



\tikzstyle{grn}=[every node/.style={circle,draw=black,outer sep=2pt,minimum
                size=15pt,text=black}, node distance=1.5cm, ->]
\tikzstyle{inh}=[>=|,-|,draw=colorinh,thick, text=black,label]
\tikzstyle{act}=[->,>=triangle 60,draw=coloract,thick,color=coloract]
\tikzstyle{inhgray}=[>=|,-|,draw=colorinhgray,thick, text=black,label]
\tikzstyle{actgray}=[->,>=triangle 60,draw=coloractgray,thick,color=coloractgray]
\tikzstyle{inf}=[->,draw=colorinf,thick,color=colorinf]
%\tikzstyle{elabel}=[fill=none, above=-1pt, sloped,text=black, minimum size=10pt, outer sep=0, font=\scriptsize,draw=none]
\tikzstyle{elabel}=[fill=none,text=black, above=-2pt,%sloped,
minimum size=10pt, outer sep=0, font=\scriptsize, draw=none]
%\tikzstyle{elabel}=[]


\tikzstyle{plot}=[every path/.style={-}]
\tikzstyle{axe}=[black,->,>=stealth']
\tikzstyle{ticks}=[font=\scriptsize,every node/.style={black}]
\tikzstyle{mean}=[thick]
\tikzstyle{interval}=[line width=5pt,red,draw opacity=0.7]
%\definecolor{lightred}{rgb}{1,0.3,0.3}

%\tikzstyle{hl}=[yellow]
%\tikzstyle{hl2}=[orange]

%\tikzstyle{every matrix}=[ampersand replacement=\&]
%\tikzstyle{shorthandoff}=[]
%\tikzstyle{shorthandon}=[]
\tikzstyle{objective}=[process,very thick,fill=yellow!50]

\tikzstyle{coopupdate}=[-stealth,decorate,decoration={zigzag,amplitude=1.5pt,post=lineto,post length=.3cm,pre=lineto,pre length=.3cm}]

\tikzstyle{labelprio}=[circle, fill=blue!30, inner sep=0pt, minimum size=13pt]
\tikzstyle{labelprio1}=[labelprio]
\tikzstyle{labelprio2}=[labelprio, fill=red!60]
\tikzstyle{labelprio3}=[labelprio, fill=orange!50]
\tikzstyle{labelprio4}=[labelprio, fill=brown!50]

\tikzstyle{labelstocha}=[rectangle, rounded corners=4pt]

\tikzstyle{andot}=[circle, fill=black, inner sep=1.2pt, draw=transparent]
\tikzstyle{anligne}=[thick]

\tikzstyle{apdot}=[andot] %[circle, fill=black, draw=black, inner sep=1]
\tikzstyle{apdotsimple}=[] %[circle, fill=black, draw=black, inner sep=1]

% Figure de résumé des liens entre les formalismes
\tikzstyle{equiv-externe}=[thick, rounded corners, draw=gray, fill=gray!10, align=center,
  inner sep=8]

% procedure, abstractions and dependencies
\newcommand{\abstr}[1]{#1^\wedge}%\text{\textasciicircum}}
\def\BS{\mathbf{BS}}
\def\aBS{\abstr{\BS}}
\def\abeta{\abstr{\beta}}
\def\aZ{\abstr{\zeta}}
\def\aY{\abstr{\xi}}

\def\beforeproc{\vartriangleleft}

\def\powerset{\wp}

\def\Sce{\mathbf{Sce}}
\def\OS{\mathbf{OS}}
\def\Obj{\mathbf{Obj}}
%\def\Proc{\mathbf{Proc}}
%\def\Sol{\mathbf{Sol}}
\newcommand{\Sol}{\mathbf{Sol}}

\usepackage{galois}
\newcommand{\theOSabstr}{toOS}
\newcommand{\OSabstr}[1]{\theOSabstr(#1)}
\newcommand{\theOSconcr}{toSce}
\newcommand{\OSconcr}[1]{\theOSconcr(#1)}

% \def\gO{\mathbb{O}}
% \def\gS{\mathbb{S}}
\def\aS{\mathcal{A}}
\def\Req{\mathrm{Req}}
%\def\Sol{\mathrm{Sol}}
\def\Cont{\mathrm{Cont}}
\def\cBS{\BS_\ctx}
\def\caBS{\aBS_\ctx}
\def\caS{\aS_\ctx}
\def\cSol{\Sol_\ctx}
\def\cReq{\Req_\ctx}
\def\cCont{\Cont_\ctx}

\def\any{\star}

% \def\gProc{\mathrm{maxPROC}}
\def\mCtx{\mathrm{maxCtx}}

\def\procs{\f{procs}}
\def\objs{\f{objs}}
\def\sat#1{\lceil #1\rceil}

\def\gCont{\f{maxCont}}
\def\lCont{\f{minCont}}
\def\lProc{\f{minProc}}
\def\gProc{\f{maxProc}}

\def\join{\oplus}
\def\concat{\!::\!}
\def\emptyseq{\varepsilon}
\def\ltw{\preccurlyeq_{\OS}}
\def\indexes#1{\mathbb{I}^{#1}}
%\def\indexes#1{\{1..|#1|\}}
\def\supp{\f{support}}
\def\w{\omega}
\def\W{\Omega}
\def\ctx{\varsigma}
\def\Ctx{\mathbf{Ctx}}
\def\mconcr{\gamma}
\def\concr{\mconcr_\ctx}
\def\obj#1#2{{#1\!\Rsh^*\!\!#2}}
\def\objp#1#2#3{\obj{{#1}_{#2}}{{#1}_{#3}}}
\def\A{\mathcal{A}}
\def\cwA{\A_\ctx^\w}
\def\cwReq{\Req_\ctx^\w}
\def\cwSol{\Sol_\ctx^\w}
\def\cwCont{\Cont_\ctx^\w}
\def\gCtx{\f{maxCtx}}
\def\endCtx{\f{endCtx}}
\def\ceil{\f{end}}

%\def\lfp{\mathrm{lfp}\;}
%\def\mlfp#1{\mathrm{lfp}\{#1\}\;}
\newcommand{\lfp}[3]{\mathbf{lfp}\{#1\}\left(#2\mapsto#3\right)}
\def\maxobjs{{\f{maxobjs}}}
\def\maxprocs{{\f{maxprocs}_\ctx}}
\def\objends{{\f{ends}}}

\def\ra{\rho}
\def\rb{\rho^\wedge}
\def\rc{\widetilde{\rho}}
\def\interleave{\f{interleave}}

\def\join{\concat}

\def\procs{\mathsf{procs}}
%\def\allprocs{\mathsf{allProcs}}
\def\allprocs{\procs}
%\def\pfp{\mathsf{pfp}}
\def\pfp{\mathsf{lst}}
\def\pfpprocs{\mathsf{pfpProcs}}
\def\bounceprocs{\mathsf{bounceProcs}}
\def\newprocs{\mathsf{newProcs}}

\def\aB{\mathcal{B}}
\def\sat#1{\lceil #1\rceil}
\def\cwB{\sat{\aB_\ctx^\w}}
\def\mycwB#1#2{\sat{\aB_{#1}^{#2}}}
\def\Bsol{\sat{\Sol^\w_\ctx}}
\def\Breq{\sat{\Req^\w_\ctx}}
\def\Bcont{\sat{\Cont^\w_\ctx}}

\def\myB{\aB^\w_\ctx}
\def\mysol{\overline{\Sol^\w_\ctx}}
\def\myreq{\overline{\Req^\w_\ctx}}
\def\mycont{\overline{\Cont^\w_\ctx}}

\begin{comment}
\def\PrioCont{\textcolor{red}{\mathrm{PrioCont}}}
\def\mypriocont{\overline{\PrioCont^\w_\ctx}}
\def\cwPrioCont{\PrioCont_\ctx^\w}
\def\Bpriocont{\sat{\PrioCont^\w_\ctx}}
\def\Sat{\PrioCont}
\def\mysat{\overline{\Sat^\w_\ctx}}
\def\cwSat{\Sat_\ctx^\w}
\def\Bsat{\sat{\Sat^\w_\ctx}}

\def\ReqSolPrio{\textcolor{blue}{\mathrm{ReqSolPrio}}}
\def\RSP{\ReqSolPrio}
\def\myrsp{\overline{\RSP^\w_\ctx}}
\def\cwRSP{\RSP_\ctx^\w}
\def\Brsp{\sat{\RSP^\w_\ctx}}
\end{comment}

\newcommand{\csState}{\mathsf{procState}}

\newcommand{\V}{V}
\newcommand{\E}{E}
\newcommand{\cwV}{\V_\ctx^\w}
\newcommand{\cwE}{\E_\ctx^\w}
%\newcommand{\VProc}{\textcolor{red}{\V_\PHproc}}
%\newcommand{\VObj}{\textcolor{red}{\V_\Obj}}
%\newcommand{\VSol}{\V_{Sol}}
%\newcommand{\VSol}{\textcolor{red}{\V_{\Sol}}}
\newcommand{\VProc}{\V \cap \PHproc}
\newcommand{\VObj}{\V \cap \Obj}
\newcommand{\VSol}{\V \cap \Sol}

\def\Bv{\sat{\cwV}}
\def\Be{\sat{\cwE}}
\def\BvProc{\textcolor{red}{\sat{\cwV}^\PHproc}}
\def\BvObj{\textcolor{red}{\sat{\cwV}^\Obj}}
%\def\BvSol{\sat{\cwV}^{Sol}}
\def\BvSol{\textcolor{red}{\sat{\cwV}^{\Sol}}}

\newcommand{\Bee}[2]{\Be^{#1}_{#2}}

%\def\mlfp#1{\f{pppf}\{#1\}}

\def\PHobjp#1#2#3{\PHobj{{#1}_{#2}}{{#1}_{#3}}}
\def\Obj{\mathbf{Obj}}
\def\powerset{\wp}
\def\gCont{\f{maxCont}}

\def\muconcr{\ell}
\def\uconcr{\muconcr_\ctx}

\begin{comment}
%\newcommand{\abstr}[1]{#1^\wedge}%\text{\textasciicircum}}
%\def\priomax{\mathsf{prio}_{max}}
\def\procs{\mathsf{procs}}
\def\allprocs{\mathsf{allProcs}}
\def\pfp{\mathsf{pfp}}
\def\pfpprocs{\mathsf{pfpProcs}}
%
\def\ctx{\varsigma}
\def\w{\omega}
%\def\aBS{\abstr{\BS}}
%
\def\Req{\mathrm{Req}}
\def\Sol{\mathrm{Sol}}
\def\Cont{\mathrm{Cont}}
\def\A{\mathcal{A}}
\def\cwA{\A_\ctx^\w}
\def\cwReq{\Req_\ctx^\w}
\def\cwSol{\Sol_\ctx^\w}
\def\cwCont{\Cont_\ctx^\w}
%
%
%
\end{comment}
%%%%%%%%%%%%%%%%%%%%%%%%%%%%%%%%%%


%%%%%%%%%%%%%%%%%%%%% DEBUT ARTICLE %%%%%%%%%%%%%%%%%%%%%%%%%%%%%%%%
\usepackage{url}
\urldef{\mailsa}\path|emna.ben-abdallah@irccyn.ec-nantes.fr|
\newcommand{\keywords}[1]{\par\addvspace\baselineskip
\noindent\keywordname\enspace\ignorespaces#1}

\begin{document}

\mainmatter  % start of an individual contribution

% The title
\title{Answer Set Programming Method for Network Completion for Time-Varying Genetic Networks modeled in Process Hitting}

% a short form should be given in case it is too long for the running head
\titlerunning{ASP Method for Network Completion for Time-Varying Genetic Networks}

% the name(s) of the author(s) follow(s) next
%
% NB: Chinese authors should write their first names(s) in front of
% their surnames. This ensures that the names appear correctly in
% the running heads and the author index.
%
\author{Emna Ben Abdallah\inst{1} \and  Tony Ribeiro \inst{2}  \and Morgan Magnin \inst{1,2} \and Katsumi Inoue \inst{1}}
%
\authorrunning{E. Ben Abdallah, T. Ribeiro, M. Magnin, K. Inoue}
% (feature abused for this document to repeat the title also on left hand pages)

% the affiliations are given next; don't give your e-mail address
% unless you accept that it will be published
\institute{LUNAM Université, \'Ecole Centrale de Nantes,
 IRCCyN UMR CNRS 6597\\ (Institut de Recherche en Communications et Cybern\'etique de Nantes),\\
  1 rue de la Noë, 44321 Nantes, France.\\
\and
National Institute of Informatics, \\
2-1-2, Hitotsubashi, Chiyoda-ku, Tokyo 101-8430, Japan.
}

%
% NB: a more complex sample for affiliations and the mapping to the
% corresponding authors can be found in the file "llncs.dem"
% (search for the string "\mainmatter" where a contribution starts).
% "llncs.dem" accompanies the document class "llncs.cls".
%

\toctitle{ASP methods for Network Completion for Time-Varying Genetic Networks}
\tocauthor{Authors' Instructions}
\maketitle


\begin{abstract}

IAlmost the models of biological networks are not robust and need some times to be revised and adapted to the new observations. A system maintains its func- tions against internal and external perturbations, leading to topological changes in the network with varying delays. To understand the resilient behaviour of biological networks, we propose novel methods. First we propose an approach to model a time- dependent asynchronous and non-deterministic networks through Process Hitting (PH) framework which is a new framework particularly suitable to model biological regulatory networks. Second we have developed a novel network completion algo- rithm for time-varying networks to analyse its behavior based on the framework of network completion. This completion aims to make the minimum amount of modifications to a given network so that the resulting network is most consistent with the observed data. We demonstrate the effectiveness of our proposed methods through computational experiments using synthetic gene expression data of the circadien clock network modeled through PH. The results indicate that our methods exhibit good performance in terms of completing and inferring gene association networks with time-varying structures.

\keywords{Answer Set Programming, Process Hitting, time-varying genetic networks, network completion}

\end{abstract}


  \section{Introduction} 

	\emph{Answer set programming} (ASP) is a form of declarative programming that has been successively used in many knowledge representation and
	reasoning tasks \cite{DBLP:journals/amai/Niemela99,Baral03,DBLP:conf/iclp/Baral08}.
	In ASP, a problem is represented by a logic program where the answer sets correspond to the solutions of the problem.
	Solving the problem is then reduced to computing stable models using answer set solvers like \emph{clasp} \cite{DBLP:conf/lpnmr/GebserKNS07a,gebser2008user}.

  % Process Hitting


\section{Process Hitting }

Definition \ref{def:PH} introduces the Process Hitting(PH) \cite{PMR10-TCSB}
which allows to model a finite number of local levels,
called \emph{processes},
grouped into a finite set of components, called \emph{sorts}.
A process is noted $a_i$, where $a$ is the sort's name,
and $i$ is the process identifier within sort $a$.
At any time, exactly one process of each sort is \emph{active},
and the set of active processes is called a \emph{state}.

The concurrent interactions between processes are defined by a set of \emph{actions}.
Each action is responsible for the replacement of one process by another of the same sort
conditioned by the presence of at most one other process in the current state.
An action is denoted by $\PHfrappe{a_i}{b_j}{b_k}$, which is read as
“$a_i$ \emph{hits} $b_j$ to make it \emph{bounce} to $b_k$”,
where $a_i$, $b_j$, $b_k$ are processes of sorts $a$ and $b$,
called respectively \emph{hitter}, \emph{target} and
\emph{bounce} of the action.
We also call a \emph{self-hit} any action whose hitter and target sorts are the same,
that is, of the form: $\PHfrappe{a_i}{a_i}{a_k}$.

The PH is therefore a restriction of synchronous automata, where each transition
changes the local state of exactly one automaton,
and is triggered by the local states of at most two distinct automata.
This restriction in the form of the actions was chosen to permit
the development of efficient static analysis methods
based on abstract interpretation \cite{PMR12-MSCS}.

\begin{definition}[Process Hitting]\label{def:PH}
  A \emph{Process Hitting} is a triple $(\PHs,\PHl,\PHa)$ where:
  \begin{itemize}
    \item  $\PHs = \{a,b,\dots\}$ is the finite set of \emph{sorts};
    \item  $\PHl = \prod_{a\in\PHs} \PHl_a$ is the set of \emph{states} where
      $\PHl_a = \{a_0,\dots,a_{l_a}\}$
      is the finite set of \emph{processes} of sort $a\in\Sigma$
      and $l_a$ is a positive integer, with $a\neq b\Rightarrow \PHl_a \cap \PHl_b = \emptyset$;
    \item  $\PHa = \{ \PHfrappe{a_i}{b_j}{b_k} \in \PHl_a \times \PHl_b^2 \mid
      (a,b) \in \PHs^2 \wedge b_j\neq b_k \wedge a=b\Rightarrow a_i=b_j\}$
      is the finite set of \emph{actions}.
  \end{itemize}
\end{definition}

\begin{example}
The figure \ref{fig:ph} represents a $\PH$ $(\PHs,\PHl,\PHa)$ with three sorts
($\PHs = \{a, b, c\}$) and:
$\PHl_a = \{a_0, a_1\}$,
$\PHl_b = \{b_0, b_1\}$,
$\PHl_z = \{z_0, z_1, z_2\}$.
\begin{figure}[ht]
\label{fig:ph} 
\centering
\begin{tikzpicture}%[font=\scriptsize]
%\path[use as bounding box] (0,-1) rectangle (4,4);

\TSort{(0,0)}{z}{3}{l}
\TSort{(2,4)}{b}{2}{t}
\TSort{(4,1)}{a}{2}{r}
\THit{b_0}{}{z_1}{.east}{z_2}
\THit{b_1}{}{z_0}{.north east}{z_2}
\THit{a_0}{}{b_1}{.south}{b_0}
\THit{a_1}{out=60,in=0,selfhit}{a_1}{.east}{a_0}

\path[bounce,bend right]
\TBounce{z_1}{}{z_2}{.south}
\TBounce{z_0}{bend right=50}{z_2}{.south east}
;
\path[bounce,bend left]
\TBounce{a_1}{}{a_0}{.north}
\TBounce{b_1}{}{b_0}{.south}
;

 \THit{z_0}{}{a_0}{.west}{a_1} 

\path[bounce,bend left]
\TBounce{a_0}{}{a_1}{.south}
;
\TState{a_0,b_0,z_1}
\end{tikzpicture}
\caption{
A PH model example with three sorts: $a$, $b$ and $z$ ($a$ is either at level 0 or 1, $b$ at either level 0 or 1 and $z$ at either level 0, 1 or 2). Boxes represent the \emph{sorts} (network components), circles represent the \emph{processes} (component levels), and the 5 \emph{actions} that model the dynamic behavior are depicted by pairs of arrows in solid and dotted lines. The grayed processes stand for the possible initial state: $\PHstate{a_1, b_0, z_1}$.
}
\end{figure}
\end{example}

A state of the networks is a set of active processes containing a single process of each sort.
The active process of a given sort $a \in \PHs$ in a state $s \in \PHl$
is noted $\PHget{s}{a}$.
For any given process $a_i$ we also note: $a_i \in s$ if and only if $\PHget{s}{a} = a_i$. The dynamic of the PH networks is satisfied thanks to the actions. Indeed, the transition from one state $s_1$ to its successor $s_2$ is done when there is a playable action (definition \ref{def:playableAction}) at $s_1$. After each transition only one sort, or one component, changes its level from one process to another.

%\begin{definition}[Action]
%\label{def:PhAction}
%An action is noted $\PHfrappe{a_i}{b_j}{b_k}$ where $a_i$ is a process
%of sort $a$ and $b_j$, $b_k$ two processes of sort $b$. When $a_i = b_j$ , such an action is %refered as a self-action and $a_i$
%is called a self-hitting process.
%\end{definition}

\begin{definition} [Playable action]
\label{def:playableAction}
Let $\PH = (\PHs,\PHl,\PHa)$ be a Process Hitting and $s \in \PHl$ a state of $PH$.
We say that the action $h = \PHfrappe{a_i}{b_j}{b_k} \in \PHa$
is \emph{playable in state $s$} if and only if
$a_i \in s$ and $b_j \in s$ (\ie $\PHget{s}{a} = a_i$ and $\PHget{s}{b}=b_j$).
The resulting state after playing $h$ in $s$
is called a \emph{successor} of $s$ and
is denoted by $(s \play h)$,
where $\PHget{(s \play h)}{b} = b_k$ and
$\forall c \in \PHs, c \neq b \Rightarrow \PHget{(s \play h)}{c}=\PHget{s}{c}$.
\end{definition}

We note that during these last years the Process Hitting framework was improved and we added new type of sorts like cooperative sorts and new actions like plural actions, actions with priority and actions with delay.

In some cases it is necessary to represent a reaction of a set of components on one component. For example in the bio-chemical reactions :$X \xrightarrow{Y} Z$ or  $X + Y \rightarrow Y + Z$, where $ X $ is a set of reactives, $ Y $ a set of catalysts and $ Z $ a set of products.% In PH network the evolution is asynchronous so we consider that $Z$ is one component. 
The plural action permits to represent this kind of reactions in PH. The plural is made up of two sets of processes of different sorts, which represent all the hitters and the bonds.

\begin{definition}[Plural action]
\label{def:pluralAction}
It is a reaction of this form: \\
 $ \PHfrappemult{A}{B} \mid A, B \in Proc \setminus \emptyset \wedge \forall q \in B, \exists p \in A, (p \neq q \wedge \PHsort(p) = \PHsort(q)) $
\end{definition}

\begin{example}
We give a simple example to represent a plurial action by a cooperation between two biological components ($x$ and $y$) in order to activate another component ($z$) and change its level from $0$ to $1$: $\{x_1, y_1, z_0 \} \rightarrow \{x_1, y_1, z_1 \} $.

\begin{figure}[ht]
\label{fig:ph-plurial} 
\centering
\begin{tikzpicture}[apdotsimple/.style={apdot}]
\TSort{(1,3)}{x}{2}{l}
\TSort{(1,0)}{y}{2}{l}
\TSort{(6,2)}{z}{2}{r}

\TActionPlur{x_1, y_1}{z_0.west}{z_1.south west}{}{4,2.5}{left}
% ex : \TActionPlur{f_1, c_0}{a_0.west}{a_1.south west}{}{3.5,2.5}{left}
\TState{x_1, y_1, z_0}

\end{tikzpicture}
\caption{
Representation of a plural action in Process Hitting network: $\{x_1, y_1, z_0 \} \rightarrow \{x_1, y_1, z_1 \} $. 
}

\end{figure}
\end{example}

In some dynamics it is crucial to have information about the delays between two events (two states in PH). The normal actions cannot show this information we just know that the state $s_2$ will be after $s_1$ in the next step but it is not possible to know how much time this transition takes time. We propose to add the delay in the action attributes which is responsable of the transition between the two states. That means that this action needs to be played during a specific time so that the system doesn't change the state (defintion \ref{def:TimedAction}).

\begin{definition}[timed action]
\label{def:TimedAction}

\end{definition}

\begin{definition}[temporised sort]
\label{def:TempSort}

\end{definition}

The PH was chosen for several reasons.
First, it is a general framework that,
although it was mainly used for biological networks,
allows to represent any kind of dynamical model,
and converters to several other representations are available (see section%~\ref{comparison}).
Although an efficient dynamical analysis already exists for this framework,
based on an approximation of the dynamics,
it is interesting to identify its limits
and compare them to the approached we present later in this paper.
Finally, the particular form of the actions in a PH model allow
to easily represent them in ASP,
with one fact per action, as described in the next section.
Other representations may have required supplementary complexity;
for instance, a labeling would be required
if actions could be triggered by a variable number of processes.

%Parler de l'asynchrone model


The rest of the report focuses on the representation of the previous definitions through ASP than we give the example of Circadian Clock network. Later we propose an approch to resolve the completion problem of PH networks with the use of ASP.


  \include{parts/circadianClock}
  % PH through ASP

\section{PH through ASP}
\label{sec:ph-asp}
\subsection{Translation of PH networks to ASP}
PH network is easy to be presented in ASP. Indeed we need only 3 predicates to define the whole network:
\texttt{"sort"} to define sorts, \texttt{"process"} for the processes and \texttt{"action"} for the network actions. We will see in example \ref{ex:asp-ph} how a PH network is defined with these predicates.

\begin{example}[Representation of a PH network in ASP]
\label{ex1:asp-ph}
The representation of the PH network of figure \ref{fig:ph} in ASP is the following:
\begin{lstlisting}
process("a", 0..1). process("b", 0..1). process("z", 0..2). %\label{ASPprocess}
sort(X) :- process(X,I) %\label{ASPsort}
action("a",0,"b",1,0). action("a",1,"a",1,0). action("b",1,"z",0,2). %\label{actions1}
action("b",0,"z",1,2). action("z",0,"a",0,1). %\label{actions2}
\end{lstlisting}
In line \ref{ASPprocess} we create the list of processes corresponding to each sort,
for example the sort \texttt{"z"} has 3 processes numbered from \texttt{0} to \texttt{2};
this specific predicate will in fact expand into the three following predicates:
\texttt{process("z", 0)}, \texttt{process("z", 1)}, \texttt{process("z", 2)}.
Line \ref{ASPsort} enumerates every sort of the network from the previous information.
Finally, all the actions of the network are defined in lines \ref{actions1} and \ref{actions2};
for example, the first predicate \texttt{action("a",0,"b",1,0)} represents the action
$\PHfrappe{a_0}{b_1}{b_0}$.
\end{example}

\begin{example}[Representation of PH network with plural-timed actions in ASP]
\label{ex2:ph-asp}
We can take the example at the figure \ref{fig:ph-plurial} and consider that the action  $\{x_1, y_1, z_0 \} \xrightarrow{D} \{x_1, y_1, z_1 \} $  has a delay "$D$" and in PH the action becomes: $x_1 \wedge \PHfrappe{y_1}{z_0}{z_1}$. So that the PH  network in ASP is:
\begin{lstlisting}
process("x", 0..1). process("y", 0..1). process("z", 0..1). %\label{ASPprocess2}
action("x",1,"y",1,"z",0,1,D). %\label{pluralaction}
\end{lstlisting}
The number of indegree in this action $i=2$, there is only 2 hitters. It is possible to have an indegree greater than $2$. We will show later that the number of the maximum indegree should be an input for in our algorithm.
\end{example}

The predicat \texttt{action} represents the ordinary actions as well as the plural actions. Indeed an ordinary action is a plural action with an indegree $1$. Moreover all the ordinary actions are a timed action with a dealy equal to $1$ unit of time. Each action need at least one step to be played  For example the action \texttt{action("a",0,"b",1,0)} is equivalent to \texttt{action("a",0,"b",1,0,1)}
So the plural-timed action is a generalized way to represent the actions in a Process Hitting network. Thus in the following part we will consider only plural-timed actions. 

\subsection{Circadian clock in ASP}
	
   \include{parts/completionPH}
   % Experiments

\section{Evaluation}


\subsection{Completion of circadian clock}

% Mettre un eemple avec une figure et les regle qu'on apprends
  \section{Conclusion and perspectives}

%Nous avons montré dans cette thèse de Master une nouvelle analyse dynamique développée. Cette analyse est applicable à une classe de modèles dits PH et vise à déterminer des propriétés des réseaux modélisés. 

%La première propriété est la recherche des états stables du réseau qu'on appelle les points fixes. Ces points représentent les états du réseau pendant lesquels, le modèle ne peut plus évoluer. Il est intéressant à les connaitre car ils bloquent l'évolution des systèmes biologiques. Nous avons développé deux méthodes, sachant que la deuxième est plus optimale, qui retournent l'ensemble de tous les points fixes du réseau. Ce point fixe n'est qu'un état du réseau traduit en ASP par un niveau pour chaque composant, autrement un processus pour chaque sorte. \\
%La deuxième propriété est une propriété qui se base sur la dynamique du réseau: l'atteignabilité. Un réseau biologique évolue dans plusieurs sens qui peuvent aboutir ou pas à des états-objectifs. Notre nouvelle approche retourne les chemins exactes aboutissant à atteindre un niveau cible d'un composant du système. Ce chemin se traduit par un ensemble de changements successives des niveaux. En PH cela se traduit par de changements successifs de processus et c'est ce que nos méthodes retournent comme résultat. Il s'avére que la méthode itérative en ASP est plus optimale que celle en ASP normal. En effet il n'est pas nécessaire de prévoir le nombre de changements pour la méthode itérative et elle retourne le résultat plus rapidement (quelques seconde pour des réseau moyennement grand).

%Une comparaison a été faite par rapport à l'existant, \textsc{PINT} et la méthode de Rocca et al. Les résultats montrent que, par rapport à \textsc{PINT}, la méthode de recherche des points fixes est efficace, mais que pour l'accessibilité, elle l'est moins que prévu. Cependant aussi notre méthode retourne de plus le chemin d'atteignabilité. Elle offre la possibilité de poser des questions plus générales par rapport à \textsc{PINT} portant sur plusieurs sortes de plus. \\
%Sachant que la présentation d'un réseau biologique en PH est simplifie le traitement et la traduction des modèles, il s'avère que notre méthode qui se base sur ce formalisme est plus efficace que d'autres méthodes développées en ASP aussi mais pour des réseau de graphes de transitions. Le cas de la méthode de Rocca qui est gourmande en temps par rapport à la notre, résultat retourné en des minutes contre un résultat affiché en quelques secondes.\\

%Nous pensons que cette approche peut également être utilisée et adaptée avec d'autres modèles tels que le modèle de Thomas, les réseaux de Petri et les modèles synchrones. Cela nécessite une traduction propre au modèle étudié ainsi qu'un traitement approprié.\\
%Parmi nos perspectives qui font partie de mon sujet de thèse, c'est d'essayer d'améliorer cette méthode en éliminant les cycles de la méthode itérative. Cela évite de tourner indéfiniment dans des boucles sans avoir un résultat affiché.\\ 

%Ensuite, nous souhaitons étendre le programme pour chercher les attracteurs. Un attracteur est un ensemble d'états à partir desquels il n'est plus possible de sortir, et donc tel que le réseau tourne indéfiniment dans ces états. Le point fixe est un cas spécial des attracteur, en effet c'est un attracteur de dimension une. Par contre, la caractérisation des attracteurs  de la dynamique,de dimension $n$, requiert une analyse des dynamiques possibles bien plus poussée que pour les points fixes.\\
%Nous visons aussi à implémenter une recherche dynamique dans le sens inverse de l'atteignabilité et poser la question: "\textit{Quels sont les états initiaux qui nous permettent d'atteindre nos objectifs?}".  La réponse à cette question est l'ensemble des états à partir des quels il existe des chemins qui activent le ou les objectif(s). C'est vrai que la méthode de Rocca et al. résolve cette problématique mais nous estimons à avoir une approche plus efficace en terme de temps et qui utilise le réseau en process hitting en non pas les graphes de transitions.

%Tous ces problématiques constituent une perspective intéressante dans le cadre du développement de techniques d'analyse statique et dynamiques des propriétés du Process Hitting.\\
  
  \bibliographystyle{plain}
  \bibliography{biblio}
    
\end{document}

