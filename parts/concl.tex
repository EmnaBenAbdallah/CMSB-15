\section{Conclusion and perspectives}
\label{sec:conclusion}

In this paper, we proposed an approach to automatically infer timed models of Process Hitting from time series data (expressed as chronograms). To do so, we implemented our algorithm in ASP. We illustrated the applicability and limits of the method through various benchmarks. This opens the way to promising applications in the connection between biologists and computer scientists. Further works will now consist in discussing the kind of information one can get on timed Process Hitting by analyzing the associated untimed model. We also plan to improve our implementation to make it robust against noisy data.  

%Nous avons montré dans cette thèse de Master une nouvelle analyse dynamique développée. Cette analyse est applicable à une classe de modèles dits PH et vise à déterminer des propriétés des réseaux modélisés. 

%La première propriété est la recherche des états stables du réseau qu'on appelle les points fixes. Ces points représentent les états du réseau pendant lesquels, le modèle ne peut plus évoluer. Il est intéressant à les connaitre car ils bloquent l'évolution des systèmes biologiques. Nous avons développé deux méthodes, sachant que la deuxième est plus optimale, qui retournent l'ensemble de tous les points fixes du réseau. Ce point fixe n'est qu'un état du réseau traduit en ASP par un niveau pour chaque composant, autrement un processus pour chaque sorte. \\
%La deuxième propriété est une propriété qui se base sur la dynamique du réseau: l'atteignabilité. Un réseau biologique évolue dans plusieurs sens qui peuvent aboutir ou pas à des états-objectifs. Notre nouvelle approche retourne les chemins exactes aboutissant à atteindre un niveau cible d'un composant du système. Ce chemin se traduit par un ensemble de changements successives des niveaux. En PH cela se traduit par de changements successifs de processus et c'est ce que nos méthodes retournent comme résultat. Il s'avére que la méthode itérative en ASP est plus optimale que celle en ASP normal. En effet il n'est pas nécessaire de prévoir le nombre de changements pour la méthode itérative et elle retourne le résultat plus rapidement (quelques seconde pour des réseau moyennement grand).

%Une comparaison a été faite par rapport à l'existant, \textsc{PINT} et la méthode de Rocca et al. Les résultats montrent que, par rapport à \textsc{PINT}, la méthode de recherche des points fixes est efficace, mais que pour l'accessibilité, elle l'est moins que prévu. Cependant aussi notre méthode retourne de plus le chemin d'atteignabilité. Elle offre la possibilité de poser des questions plus générales par rapport à \textsc{PINT} portant sur plusieurs sortes de plus. \\
%Sachant que la présentation d'un réseau biologique en PH est simplifie le traitement et la traduction des modèles, il s'avère que notre méthode qui se base sur ce formalisme est plus efficace que d'autres méthodes développées en ASP aussi mais pour des réseau de graphes de transitions. Le cas de la méthode de Rocca qui est gourmande en temps par rapport à la notre, résultat retourné en des minutes contre un résultat affiché en quelques secondes.\\

%Nous pensons que cette approche peut également être utilisée et adaptée avec d'autres modèles tels que le modèle de Thomas, les réseaux de Petri et les modèles synchrones. Cela nécessite une traduction propre au modèle étudié ainsi qu'un traitement approprié.\\
%Parmi nos perspectives qui font partie de mon sujet de thèse, c'est d'essayer d'améliorer cette méthode en éliminant les cycles de la méthode itérative. Cela évite de tourner indéfiniment dans des boucles sans avoir un résultat affiché.\\ 

%Ensuite, nous souhaitons étendre le programme pour chercher les attracteurs. Un attracteur est un ensemble d'états à partir desquels il n'est plus possible de sortir, et donc tel que le réseau tourne indéfiniment dans ces états. Le point fixe est un cas spécial des attracteur, en effet c'est un attracteur de dimension une. Par contre, la caractérisation des attracteurs  de la dynamique,de dimension $n$, requiert une analyse des dynamiques possibles bien plus poussée que pour les points fixes.\\
%Nous visons aussi à implémenter une recherche dynamique dans le sens inverse de l'atteignabilité et poser la question: "\textit{Quels sont les états initiaux qui nous permettent d'atteindre nos objectifs?}".  La réponse à cette question est l'ensemble des états à partir des quels il existe des chemins qui activent le ou les objectif(s). C'est vrai que la méthode de Rocca et al. résolve cette problématique mais nous estimons à avoir une approche plus efficace en terme de temps et qui utilise le réseau en process hitting en non pas les graphes de transitions.

%Tous ces problématiques constituent une perspective intéressante dans le cadre du développement de techniques d'analyse statique et dynamiques des propriétés du Process Hitting.\\