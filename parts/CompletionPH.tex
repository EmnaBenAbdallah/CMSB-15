% Completion of PH network (Circadian clock)

\section{Completion of PH networks}

\subsection{Asynchronous and non-deterministic networks}

\subsection{Algorithm}

In this section we propose the {\it ASPH-Completion} algorithm to complete Process Hitting networks.
This algorithm takes as input a Process Hitting and a chronogram of genes evolutions of this network.
Knowing the genes influences (or assuming all possible influences),
this algorithm will complete the input network by adding the delayed actions that could have realized the changes observed in the chronogram.
Algorithm \ref{alg:ASPHC} show the pseudo code of our completion algorithm.
It first consider the gene changes, to generates all possible actions that can realize each change.
Then, it consider the time steps where there is no changes to remove the actions that would made one at this moment.

\begin{algorithm}
	\caption{ASPH-Completion($PH,Chronogram,Influences,indegree$)}
	\label{alg:ASPHC}
	\begin{itemize}
		\item INPUT: a Process Hitting $PH$, a chronogram $C$ of the genes evolution of $PH$, the genes influences and a maximal action indegree $i$.

		%\item Step 1: Sample the chronogram for each gene at each time step.
		\item Let $PH' := PH$
		\item Step 1: For each time where a gene $G$ changes its value from $val$ to $val'$ in $C$:

		\begin{itemize}
			\item[-] 1.1: Let $D$ be the delay since the last time a gene has changed.
			\item[-] 1.2: Generate all actions with delay $D$ which involve all subsets of size $i$ of the genes $X_1, \ldots, X_n$ having an influence on $G$:
			$$A := action(X_1,Val_1,\ldots, X_n,Val_n, G, val, val', D)$$
			\item Add all action $A$ in $PH'$
		\end{itemize}

		\item Step 2: For each time step $t$ where there is no gene change
		\begin{itemize}
			\item[-] Remove from $PH'$ all actions that can be fire at $t$
		\end{itemize}
		\item Step 3: Merge each action with the same hitters, $X_1,Val_1,\ldots, X_n,Val_n$ and the same target, $G, val, val'$, into one action where the delay is the average.
		\item OUTPUT:  a completed Process Hitting that realize the chronogram.
	\end{itemize}
	\textcolor{red}{\\Le Step 3 est formellement unsafe !!!}
\end{algorithm}

\begin{theorem}[Correctness and Completeness]
	\label{th:correct}
	Let $PH$ be a Process Hitting, $C$ be a chronogram of the genes of $PH$ and $A$ be the set of actions of $PH$ whose realized the chronogram $C$.
	Let $PH'$ be a Process Hitting and $A'$ be the set of actions of $PH'$ such that $A' \subseteq A$.
	Given $PH'$ and $C$ as input, {\it ASPH-Completion} (without step 3) will output a process hitting $PH''$,
	$A''$ the set of actions of $PH''$ such that $A \subseteq A''$ and $A''$ can realize $C$.
\end{theorem}

	As stated by Theorem \ref{th:correct}, {\it ASPH-Completion} is correct and complete.
	Since it generates all actions that can realized each gene change (step 1.2),
	knowing the genes influences (or assuming all possible influences),
	there is no action $a \in A$ that realized $C$ which is not generated by the algorithm.
	Also, each gene change can be realized by one of the action generated at step 1.2,
	so that the completed Process Hitting outputted by the algorithm can realize the input chronogram $C$.
	All action of the outputted Process Hitting $PH'$ are consistent with the input chronogram $C$.

\begin{theorem}[Complexity]
	\label{th:complexity}
	Let $PH$ be a Process Hitting, $S$ be the number of sorts of $PH$ and $P$ be the maximal number of processes of a sort of $PH$.
	Let $C$ be a chronogram of the genes of $PH$ over $T$ units of time.
	The complexity of completing $PH$ by generating actions from the observations of $C$ with {\it ASPH-Completion} belongs to $O(T*P^{S+1})$.
	\begin{proof}
		% Memory
		Let $i$ be the maximal indegree of an action in $PH$, $0 \leq i \leq P$.
		Let $p$ be a process of $PH$ and $n$ be the number of sorts that can influence $p$.
		There is $i^S$ possible combinations of those process that can hit $p$, each of those can form an action.
		Since there is $P$ process, there are $P * i^S$ possibles actions, thus the memory of our algorithm is bound by $O(P * i^S)$,
		which belongs to $O(P^{S+1})$ since $0 \leq i \leq P$.
	
		% Run time
		Sampling the chronogram (step 1) is linear in the number of time step and then bound by $O(T)$.
		At each time step, atmost one gene can change its value in the chronogram $C$.
		Thus atmost $i^S$ actions can be produced at each time step, the complexity is then bound by $O(T * i^S)$.
		Since $0 \leq i \leq P$, the complexity of generating actions from the observations of $C$ (Step 2) belongs by $O(T * P^S)$.
		Merging the actions (step 3) is linear in the number of actions and then bound by $O(P^{S+1})$.
		So, finally, the complexity of completing $PH$ by learning actions from the observations of $C$ with our algorithm is $O(T + T * i^S + P^{S+1})$,
		which belongs to $O(T * P^{S+1})$
		$\qed$
	\end{proof}
\end{theorem}

%dire que les action du PH de base sont aussi merge dans le step 3. Et qu'elle peuvent etre donnee de base dans l'apprentissage
% Dire que les influences doivent etre donnees, ou alors on genere toutes les influences possibles.


