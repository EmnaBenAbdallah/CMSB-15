% Completion of PH network (Circadian clock)

\section{Completion of PH networks}

\subsection{Asynchronous and non-deterministic networks}

\subsection{Formalization}

\begin{definition}[Applicability]
	Let $C$ be a chronogram and $a$ be an action of a Process Hitting $PH$ such that $a = action(X_1,Val_1,\ldots, X_n,Val_n, G, val, val', D)$.
	Let $t$ be a time step of $C$, the action $a$ is {\it applicable} at $t$,
	if there is a gene change at $t-D$, no gene change during the interval $[t-D,t]$ and all $X_i$ has the value $Val_i$ from $t-D$ to $t$.
\end{definition}

% TODO: Préciser quelque part que si au moins une action est applicable a un instant $t$, il doit y avoir un changement

\begin{definition}[Consistency]
	Let $C$ be a chronogram and $a$ be an action of a Process Hitting $PH$ such that $a = action(X_1,Val_1,\ldots, X_n,Val_n, G, val, val', D)$.
	The action $a$ is {\it consistent} with $C$, if there is a gene change at each time step $t$ of $C$ where $a$ can be applied.
\end{definition}

\subsection{Algorithm}

In this section we propose two algorithms to complete Process Hitting networks.
Both algorithms takes as input a Process Hitting and a chronogram of genes evolutions of this network.
Knowing the genes influences (or assuming all possible influences),
these algorithms will complete the input network by adding the delayed actions that could have realized the changes observed in the chronogram.
The first algorithm we propose is a complete approach:
assuming that all input observations are perfect it will correctly revise the input Process hitting by adding new actions and removing existing ones.
Algorithm \ref{alg:ASPHC} show the pseudo code of our completion algorithm.
It first consider the gene changes, to generates all possible actions that can realize each change.
At this step, new actions can be added to the input network to perform the changes observed that no current action in the network could perform.
Then, it consider the time steps where there is no changes to remove the actions that would made one at this moment.
Here, both the actions generated and existing ones that are not consistent with the observation are removed from the network.

\begin{algorithm}
	\caption{PH-Completion($PH,Chronogram,Influences,indegree$)}
	\label{alg:PHC}
	\begin{itemize}
		\item INPUT: a Process Hitting $PH'$, a chronogram $C$ of the genes evolution of a Process Hitting $PH$, the genes influences and a maximal action indegree $i$.

		%\item Step 1: Sample the chronogram for each gene at each time step.
		\item Let $PH'' := PH'$
		\item Step 1: For each time where a gene $G$ changes its value from $val$ to $val'$ in $C$:

		\begin{itemize}
			\item[-] 1.1: Let $D$ be the delay since the last time a gene has changed.
			\item[-] 1.2: Generate all actions with delay $D$ which involve all subsets of size $i$ of the genes $X_1, \ldots, X_n$ having an influence on $G$:
			$$A := action(X_1,Val_1,\ldots, X_n,Val_n, G, val, val', D)$$
			\item Add all action $A$ in $PH''$
		\end{itemize}

		\item Step 2: For each time step $t$ where there is no gene change
		\begin{itemize}
			\item[-] Remove from $PH''$ all actions that are applicable at $t$
		\end{itemize}
		\item OUTPUT: $PH''$ a completed Process Hitting that realizes the chronogram.
	\end{itemize}
\end{algorithm}

\begin{theorem}[Completeness and Correctness]
	\label{th:correct}
	Let $PH$ be a Process Hitting, $C$ be a chronogram of the genes of $PH$ and $A$ be the set of actions of $PH$ whose realized the chronogram $C$.
	Let $PH'$ be a Process Hitting and $A'$ be the set of actions of $PH'$ such that $A' \subseteq A$.
	Given $PH'$ and $C$ as input, Algorithm \ref{alg:PHC} is complete and correct: it will output a process hitting $PH''$,
	with $A''$ the set of actions of $PH''$ such that $A \subseteq A''$ ($A''$ can realize $C$) and all action $a \in A''$ are consistent with $C$.
	\begin{proof}
	Lets suppose that the algorithm is not complete, then there is an action $a \in A$ that realized $C$ and $a \not \in A''$.
	After step 1.2, $PH''$ contains all actions that can realized each gene change.
	Here there is no action $a \in A$ that realized $C$ which is not generated by the algorithm, so $a \in A''$.
	Then it implies that at step 2, the action $a$ is removed from $A''$.
	It means that there is a time step $t$ where $a$ is applicable, but in the chronogram $C$ there is no gene change at $t$.
	But since $a \in A$, a change should have occurs at $t$, this is a contradiction.
	So there is no $a \in A$ that could be remove from $A''$ at step 2, the algorithm is complete.
	$\qed$
	
	The algorithm is also correct and this conclusion is trivial from the operation of step 2.
	If it was not correct, then there is an action $a \in A''$ that is not consistent with $C$.
	But all such action are removed at step 2.
	So that the algorithm output is correct.
	
	All action of the outputted Process Hitting $PH'$ are consistent with the input chronogram $C$.
	\end{proof}
\end{theorem}

The second algorithm we propose is an approximated approach:
assuming that input observations can be noisy it will revise the input Process hitting by adding action and changing the delay of existing ones.
Algorithm \ref{alg:ASPHC_ap} show the pseudo code of our approximated algorithm.
Like in Algorithm \ref{alg:ASPHC}, it generates all possible actions that can realize each change observed.
But then, since observation are not perfect, we cannot safely removed non-consistent actions.
Here, we just merge actions by replacing the one that only differ by there delay by one action where its delay is the average.
The intuition is that, in practice, if there is enougth observation, the delay of those actions should tend to the real value.

\begin{algorithm}
	\caption{PH-Completion($PH,Chronogram,Influences,indegree$)}
	\label{alg:PHC_ap}
	\begin{itemize}
		\item INPUT: a Process Hitting $PH$, a chronogram $C$ of the genes evolution of $PH$, the genes influences and a maximal action indegree $i$.

		%\item Step 1: Sample the chronogram for each gene at each time step.
		\item Let $PH' := PH$
		\item Step 1: For each time where a gene $G$ changes its value from $val$ to $val'$ in $C$:

		\begin{itemize}
			\item[-] 1.1: Let $D$ be the delay since the last time a gene has changed.
			\item[-] 1.2: Generate all actions with delay $D$ which involve all subsets of size $i$ of the genes $X_1, \ldots, X_n$ having an influence on $G$:
			$$A := action(X_1,Val_1,\ldots, X_n,Val_n, G, val, val', D)$$
			\item Add all action $A$ in $PH'$
		\end{itemize}
		
		\item Step 2: Merge each action with the same hitters, $X_1,Val_1,\ldots, X_n,Val_n$ and the same target, $G, val, val'$, into one action where the delay is the average.
		\item OUTPUT:  a completed Process Hitting that realize the chronogram.
	\end{itemize}
\end{algorithm}

\begin{theorem}[Complexity]
	\label{th:complexity}
	Let $PH$ be a Process Hitting, $S$ be the number of sorts of $PH$ and $P$ be the maximal number of processes of a sort of $PH$.
	Let $C$ be a chronogram of the genes of $PH$ over $T$ units of time, such that $c$ is the number of gene change of $C$.
	The complexity of completing $PH$ by generating actions from the observations of $C$ with Algorithm \ref{alg:PHC} belongs to $O(c*i^S + (T-c) * P * i^S)$ that is bound by
	$O(T^2*P^{S+2})$.
	The complexity of completing $PH$ by generating actions from the observations of $C$ with Algorithm \ref{alg:PHC_ap} belongs to $O(c*i^S + (T*P*i^S)^2)$ that is bound by
	$O(T*P^S + T^2*P^{2S+2})$.
	\begin{proof}
		% Memory
		Let $i$ be the maximal indegree of an action in $PH$, $0 \leq i \leq P$.
		Let $p$ be a process of $PH$ and $n$ be the number of sorts that can influence $p$.
		There is $i^S$ possible combinations of those process that can hit $p$, each of those can form an action.
		There is $P$ process and atmost $T$ possibles delay, so that there are $T* P * i^S$ possibles actions, thus the memory of our algorithm is bound by $O(T * P * i^S)$,
		which belongs to $O(T*P^{S+1})$ since $0 \leq i \leq P$.
	
		% Run time
		%At each time step, atmost one gene can change its value in the chronogram $C$.
		Atmost $i^S$ actions can be produced for each gene change, the complexity of step 1 is then $O(c * i^S)$.
		In step 2 of Algorithm \ref{alg:PHC} at each time step where there is no gene change, it checks all actions generated so far.
		Since there is $T * P * i^S$ possibles actions, this operation complexity is $O( (T-c) * T * P * i^S)$.
		Then the complexity of Algorithm \ref{alg:PHC} is $O(c*i^S + (T-c) * T * P * i^S)$.
		Since $0 \leq i \leq P$ and $0 \leq c \leq T$ the complexity belongs to $O(T*P^S + T^2 * P^{S+1})$ and is bound by $O(T^2*P^{S+2})$.
		
		%Sampling the chronogram (step 1) is linear in the number of time step and then bound by $O(T)$.
		In Algorithm \ref{alg:PHC_ap},
		merging the actions at step 2 is polynomial in the number of actions and the complexity of this operation belongs to $O((T*P*i^S)^2)$.
		The complexity of this algorithm belongs to $O(c*i^S + (T*P*i^S)^2)$.
		Since $0 \leq i \leq P$ and $0 \leq c \leq T$ the complexity of Algorithm \ref{alg:PHC_ap} is bound by $O(T*P^S + (T*P*P^S)^2) = (T*P^S + T^2*P^{2S+2})$.
		$\qed$
	\end{proof}
\end{theorem}


%dire que les action du PH de base sont aussi merge dans le step 3. Et qu'elle peuvent etre donnee de base dans l'apprentissage
% Dire que les influences doivent etre donnees, ou alors on genere toutes les influences possibles.


