% PH through ASP

\section{PH through ASP}
\label{sec:ph-asp}
\subsection{Translation of PH networks to ASP}
PH network is easy to be presented in ASP. Indeed we need only 3 predicates to define the whole network:
\texttt{"sort"} to define sorts, \texttt{"process"} for the processes and \texttt{"action"} for the network actions. We will see in example \ref{ex:asp-ph} how a PH network is defined with these predicates.

\begin{example}[Representation of a PH network in ASP]
\label{ex1:asp-ph}
The representation of the PH network of figure \ref{fig:ph} in ASP is the following:
\begin{lstlisting}
process("a", 0..1). process("b", 0..1). process("z", 0..2). %\label{ASPprocess}
sort(X) :- process(X,I) %\label{ASPsort}
action("a",0,"b",1,0). action("a",1,"a",1,0). action("b",1,"z",0,2). %\label{actions1}
action("b",0,"z",1,2). action("z",0,"a",0,1). %\label{actions2}
\end{lstlisting}
In line \ref{ASPprocess} we create the list of processes corresponding to each sort,
for example the sort \texttt{"z"} has 3 processes numbered from \texttt{0} to \texttt{2};
this specific predicate will in fact expand into the three following predicates:
\texttt{process("z", 0)}, \texttt{process("z", 1)}, \texttt{process("z", 2)}.
Line \ref{ASPsort} enumerates every sort of the network from the previous information.
Finally, all the actions of the network are defined in lines \ref{actions1} and \ref{actions2};
for example, the first predicate \texttt{action("a",0,"b",1,0)} represents the action
$\PHfrappe{a_0}{b_1}{b_0}$.
\end{example}

\begin{example}[Representation of PH network with plural-timed actions in ASP]
\label{ex2:ph-asp}
We can take the example at the figure \ref{fig:ph-plurial} and consider that the action  $\{x_1, y_1, z_0 \} \xrightarrow{D} \{x_1, y_1, z_1 \} $  has a delay "$D$" and in PH the action becomes: $x_1 \wedge \PHfrappe{y_1}{z_0}{z_1}$. So that the PH  network in ASP is:
\begin{lstlisting}
process("x", 0..1). process("y", 0..1). process("z", 0..1). %\label{ASPprocess2}
action("x",1,"y",1,"z",0,1,D). %\label{pluralaction}
\end{lstlisting}
The number of indegree in this action $i=2$, there is only 2 hitters. It is possible to have an indegree greater than $2$. We will show later that the number of the maximum indegree should be an input for in our algorithm.
\end{example}

The predicat \texttt{action} represents the ordinary actions as well as the plural actions. Indeed an ordinary action is a plural action with an indegree $1$. Moreover all the ordinary actions are a timed action with a dealy equal to $1$ unit of time. Each action need at least one step to be played  For example the action \texttt{action("a",0,"b",1,0)} is equivalent to \texttt{action("a",0,"b",1,0,1)}
So the plural-timed action is a generalized way to represent the actions in a Process Hitting network. Thus in the following part we will consider only plural-timed actions. 

\subsection{Circadian clock in ASP}
	